\documentclass{article}
\usepackage{graphicx}
\usepackage{amssymb,amsmath}
\usepackage{hyperref}
\begin{document}

\tableofcontents

\section{Meeting 11-20-13}
\subsection{Formalism}
\subsection{Event selection}
\subsection{R2 Extraction method}
\subsection{Notes on current observations}


% \section{QCD Lagrangian}
% \subsection{Local $SU(3)_{color}$ Symmetry}
% When the Euler Lagrange equations is applied to following lagrangian, the Dirac equation is obtained for each of the 6 fields, indexed by 'f' (to be identified with light quark flavours later):

% \begin{equation} \label{eq:Free Dirac L}
% \mathcal{L} = \sum_{f=1}^3 i \bar{q}_{f} {\gamma}^{\mu} {\partial}_{\mu} {q}_{f} + m_{f} \bar{q}_{f} {q}_{f} 
% \end{equation}

% To formulate the theory of QCD, which is a gauge theory of strong interactions, the Lagrangian is required to be invariant  under local transformations brought about by the SU(3) Group. The SU(3) group has 8 ($3^{2} - 1$) generators, $T_{a}$

% \[ T_{a} = \frac{1}{2}{\lambda}_{a},   a = 1, ...8 \]

% In its fundamental representation, the 8 ${\lambda}_{a}$ matrices (Gell-Mann matrices) are 3x3 matrices. The generators produce a set of 3x3 unitary matrices that effect rotations in this 8 dimensional space (one can think of the 8 Gell-Mann matrices as being associated with the 8 orthogonal vectors that form the space). Two of the Gell-Mann matrices are diagonal (${\lambda}_{3}, {\lambda}_{8}$) and have simultaneous eigenstates. Without delving into Group Theoretical arguments, for now, let me state that the eigenvectors of the diagonal generators are used as a basis to represent the 'objects' in this 8 dimensional space on which the rotation matrices of the group act. Thus, a general object $\psi$ can be constructed as

% \[\psi = a \begin{pmatrix} 0\\ 0\\ 1 \end{pmatrix} + b \begin{pmatrix} 0 \\1 \\0 \end{pmatrix} + c \begin{pmatrix} 1 \\0 \\0 \end{pmatrix} = \begin{pmatrix} a\\ b\\ c \end{pmatrix}\]

% where a, b, c are the coefficients of the eigenvectors $\begin{pmatrix} 0\\ 0\\ 1 \end{pmatrix}$, $\begin{pmatrix} 0\\ 1\\ 0 \end{pmatrix}$ and $\begin{pmatrix} 1\\ 0\\ 0 \end{pmatrix}$.

% A general rotation in the SU(3) space can now be written as:

% \[\psi \Rightarrow {\psi}'  = e^{{\lambda}_{a}{\theta}_{a}(x)}\psi\]

% We now wish to incorporate this formalism in the $\mathcal{L}$ and require its invariance under rotations in SU(3) space. In order to do so, we assign to each field, ${q}_{f}$, three additional degrees of freedom:

% \[{q}_{f} = \begin{pmatrix} r\\ g\\ b \end{pmatrix} \]

% 'r', 'g' and 'b' can be indentified with coefficients 'a', 'b' and 'c' noted earlier. Only in this case, they are dirac field variables and this formalism can be viewed as the field having degrees of freedom (d.o.f) in SU(3) space at every space-time point. Such d.o.f are referred to as internal d.o.f

% We now require that the $\mathcal{L}$ be invariant under local transformations, $e^{-i{\lambda}_{a}{\theta}_{a}(x)}\psi$.  The lagrangian in its orginal form is not invariant and will need modification. For the purposes of this document, it will be sufficient to state that the following modified lagrangian will be invariant:

% \begin{equation} \label {eq:QCD color L}
% \mathcal{L} = \sum_{f=1}^3 i \bar{q}_{f} ({\gamma}^{\mu} \mathcal{D}_{\mu} - m_{f}){q}_{f}  + Gluonic  d.o.f 
% \end{equation}

% [Skip details of introducing gauge fields by way of making the derivate covariant, since the details are not important right now. The point is to state that the invariance of the lagrangian under local transformations of the field variable results in 'couplings' of the fields we started out with , to the 'gauge' fields. The gauge fields are self interacting in this theory.]

% To summarize, the requirement of local gauge invariance, brings about interactions between the colored fermionic fields and the gauge fields. [For completeness, mention the self coupling of the gauge fields]

% \subsection{Global Symmetries}
% According to Noether's theorem, 'all symmetries lead to conserved quantities'. It goes beyond the scope and perhaps even the point of this document, to prove that the conserved quantities are the eigenvalues of the generator of the group used to make the symmetry transformation. For example, $U(1)_{em}$ invariance of the QED lagrangian, leads to the conservation of charge, since 'Q' (charge operator) is the generator of this group. 

% [Proof can be obtained by way of writing down the conserved current. From the expression of the conserved current, a conserved 'charge' can be obtained by integrating the current for the $\mu = 0$ space-time index. This 'charge' can be shown to be independent of time, and thus commutes with hamiltonian].


% While sometimes the generators of the group can be identified as 'physical obervables' and hence as conserved quantities that we can verify through experiment, such is not always the case(?). At a deeper level, they should be identified with producing rotations matrices that mix up the degenerate multipltets of the Hamiltonian (with which they commute for symmetrical transformations).

% For example, I am not sure if we can associate the generators of rotations in the $SU(3)_{color}$ space with conserved observables, but these generators commute with the Hamiltonian. In Group Theory language, we can form irreducible representations of different dimensions from the fundamental representation of $SU(3)_{color}$. (think of spin and how the fundamental representation is combined there to obtain the singlet and triplet. The latter are irreducible representations of dimensions 0 and 3 respectively). Using the generators of $SU(3)_{color}$, we can form matrices, particular representations of which can be rotate/transform between different objects of irredubile representation (in the spin triplet state for example, we can form matrices that can use to transform between each of the 3 states). Since the generators commute with H, all these states will have the same eigenvalues and thus be degenerate eigenstates of H. 
 
% [Write a note on Noether's theorem and conserved currents, where conserved currents lead to conserved 'Charges' (by performing the $\mu = 0$ integration of the conserved currents). For each generator of the group, there is a conserved 'Charge' i.e. [H, $Q^{a}$] = 0, where a = index over number of generators. Roughly
% \[ Q^{a} \equiv G^{a}  \]

% where G $\equiv$ Generator. 


% Now that there is an operator that commutes with H $\rightarrow$ degeneracy. The number of degenerate states depends on the particular representation of the group employed. Clarify by example of $SU(2)_{isospin}$:

% If pion/neutron, then working in the fundamental representation of $SU(2)_{isospin}$, where group elements are 2x2 matrices \& the eigenvectors form a doublet. If $\pi^{+}/\pi^{-}/\pi^{0}$, then working in 'irredicible' representation where the group elements are 3x3 matrices \& eigenvectors are triplets. Additionally, all members of the degenerate multiplet can be accessed by appropriate 'rotations'. For example, if one begins with on object that is the $\pi^{+}$, then there is 3x3 matrix, through which it can be rotated to obtain the $\pi^{-}$.]

% (?)It seems that when introducing concept of symmetries, the fundamental representation of the group is used to represent the fundamental d.o.f ($SU(3)_{color}$ = colored quarks, $SU(3)_{flavor}$ = flavored quarks) Then, naively, if one were to combine fundamental representations to obtain irreducible representations of different dimensionalities (example $SU(2)_{spin}, \frac{1}{2} \otimes \frac{1}{2} = 0 \oplus 1 $ ), then the fundamental d.o.f are being combined to form composite states (Fundamental $SU(3)_{flavor}$ represenations combined to form mesons and baryons).

% To discuss global symmetries, I will drop the exlpicit reference to the color degrees of freedom. For example, when I write $q_{f}$, it should be understood that each quark flavour has additional three degrees of freedom, namely the colored quark fields. However, those degrees of freedom are not relevant when discussing global symmetries, as will be demonstrated below.

% \subsubsection{$U(1)_{V}$}
% Since the fundamental representation is a singlet, I am not sure if this symmetry even leads to different irreducible representations. The conserved currents of this symmetry, hence the conserved charges can be identified with conserved baryon number or stated another way, strong interactions do not change the flavour. I have summarized this in the summary section. For the purposes of this document, I wanted to look into group structure that was not simple like a U(1).
% \subsubsection{$SU(3)_{V}$/$SU(3)_{flavour}$}
% In the '50s \& 60's quite a few hadrons were discovered [be more precise here], that could be organized if represented in multiplets of irreducible representations formed from combinations of the fundamental representation of the SU(3) group. If we identify the following triplet as the fundamental representation of the $SU(3)_{flavour}$ group (the flavour tag will be evident)

% \[q = \begin{pmatrix} q_{u}\\ q_{d}\\ q_{s} \end{pmatrix} \]

% then, the combination of the above yields irreducible representations:

% \[ qqq states: 3 \otimes 3 \otimes 3 \otimes = 1 \oplus 8 \oplus 8 \oplus 10 \]
% \[ q\bar{q} states:  3 \otimes \bar{3} = 8 \oplus 1 \]

% If the lagrangian has this symmetry, then within each multiplet, each of the objects can be identified with a hadron. All such hadrons in the multiplet should have the same mass. [Note no comment has been made on what the mass should be for each multiplet and that is because the H of QCD is not really well understood. We only observe hadrons and record their masses. Thus the deeper questions are still not answered, what is H? Why different multiplets have different masses. Some questions i.e. the difference in mass of the pseudoscalar and vector meson octet can be ascribed to Dynamic Chiral Symmetry Breaking - next section]

% But why are the masses within a multiplet not equal? As noted earlier, if indeed, $SU(3)_{flavour}$ was an exact symmetry, it would lead to conserved charges ($\equiv$ conserved Generators i.e. $\lambda_{a}$ matrices), hence to irreducible representations, the objects of which should have the same eigenvalue for the Hamiltonian. For example, the 8 baryons in the two octets [where is the 2nd octet?] should have the same mass. [Look over any differences in spectrum that could be attributed to EM interactions - in general look over hadron wavefunctions in more detail]

% Given that the masses of hadrons in each irreducible representation do tend to be closer to each other, one can think of the SU(3) symmetry to be 'approximately' realized or that  the symmetry is partially broken. To express this approximate symmetry, it is easiest to use the lagrangian formulation.

% If we write the lagrangian of \ref{eq:QCD color L}, by putting the quarks of different flavours (u, d and s) inside an $SU(3)_{flavour}$ multiplet, as

% \[q = \begin{pmatrix} q_{u}\\ q_{d}\\ q_{s} \end{pmatrix} \]

% and put the different quark flavour masses inside a 'mass' matrix (and for the time being assume that all the flavours masses are equal) 

% \[M = \begin{pmatrix} m_{u}&0&0\\ 0&m_{d}&0\\ 0&0&m_{s} \end{pmatrix}  =  \begin{pmatrix} m&0&0\\ 0&m&0\\ 0&0&m \end{pmatrix}\]

% Unless explcitly stated, m $\equiv$ bare mass, which is a parameter of the lagrangian. (Actually, the Standard Model's Electroweak sector does not really explicity contain mass terms, since they break the electroweak symmetry. For now it is hypothesised that this symmetry is sponatneously broken due to the 'Higgs' sector. Parameters from the higgs sector result in mass like terms for the fermion fields i.e. the masses depend on the parameters of the higgs sector. My understanding is not thorough here, but I wanted to state this for completeness)

% We can now re-write the lagrangian of \ref{eq:QCD color L} as (I am not going to write the gluonic d.o.f explicitly since they are not important in these discussions)

% \begin{equation} \label{eq:QCD flavour L}
% \mathcal{L} = \bar{q} ({\gamma}^{\mu} \mathcal{D}_{\mu} - M)q  + Gluonic  d.o.f
% \end{equation}

% If all the masses are equal, then the above lagrangian can be shown to be invariant. If the masses are different, as is the case, then the lagrangian is no longer invariant and the starting point already has a slightly broken symmetry. But such mass differences are relatively small compared to the mass scale of the hadrons (~1 GeV), hence this slighly 'broken' symmetry can be used as a starting point.

% Thus, this global symmetry of the strong interaction is an approximate symmetry, that helps in orgazing hadrons and concluding that the dynamics are governed [mostly?] by the an interaction under which, the multiplets should have had the same mass (if there was an exact symmetry). [At the time this symmetry was proposed to explain the hadron spectrum, there was no reason to believe that the hadrons were actually made up of quarks, and thus there was perhaps no reason to believe that it was the strong interaction that bound the quarks. Quarks had to be confirmed experimentally - look to roadmap in Evan's proposal. Indentify where exactly this interaction related to the fundamental strong interaction. Should perhaps organize symmetries by time line]

% \subsubsection{$SU(3)_{L} \otimes SU(3)_{R}$}
% The approximate flavour symmetry did pave the way forward in probing and making sense of the hadrons. But clearly, no fundamental understanding was reached i.e. [at one stage] it was not even confirmed that, indeed quarks are present inside hadrons. Ever since quarks were detected, understanding their fundmental interaction in the low energy regime is one of the areas of active research. For example, we still do not understand the inter quark potential that binds them inside nucleons. 

% Nevertheless, we did have somewhat of a systematic way to probe nucleons. Once quarks were identified as fermions, apart from the spatial \& flavour parts of the wave function, spin was taken into account. To keep the wave function antisymmetric, later quarks were required to have 'color'. There is ofcourse the orbital angular momentum involved. As a composite object, we now had properties of hadrons we could probe and empirically further our understanding and continue to try and relate it the fundamental $\mathcal L$ of the theory. [Further discussion from here should come later and lead up to why we are using EM probes now, after having exhausted hadronic probes]

% Now if we now set the 'mass' matrix M = 0, we have in this of limit of massless quarks, another global symmetry of the lagrangian, Chiral Symmetry. This again is an ok representation to begin with since the observed hadrons are so much heavier compared to the bare masses of the quarks that we believe to 'constitute' the hadron. Here the quarks are organized accoring to their Chirality i.e. 

% \[q_L = \begin{pmatrix} q_{u,L}\\ q_{d,L}\\ q_{s,L} \end{pmatrix}  q_R = \begin{pmatrix} q_{u,R}\\ q_{d,R}\\ q_{s,R} \end{pmatrix}\]

% and are individually rotated under the action of the $SU(3)_{L} \& SU(3)_{R}$ groups respectively.  The lagrangian can be written as:

% \begin{equation} \label{eq:Chiral lagrangian}
% \mathcal{L} = \bar{q}_{L} ({\gamma}^{\mu} \mathcal{D}_{\mu}){q}_{L} +  \bar{q}_{R} ({\gamma}^{\mu} \mathcal{D}_{\mu}){q}_{R} - (\bar{q}_{R}M{q}_{L} + \bar{q}_{L}M{q}_{R}) + Gluonic  d.o.f
% \end{equation}

% The conserved currents are:

% \begin{eqnarray} 
% L^{{\mu},a} = \bar{q}_{L}{\gamma}^{\mu}\frac{{\lambda}^{a}}{2}{q}_{L} \label{eq:L currents}\\
% R^{{\mu},a} = \bar{q}_{R}{\gamma}^{\mu}\frac{{\lambda}^{a}}{2}{q}_{R} \label{eq:R currents} 
% \end{eqnarray}

% Naively, one would proceed by combining the fundamental representations of $SU(3)_{L} \& SU(3)_{R}$ and then obtain different dimensional irreducible representations. These representations can then be verified with real life obervations of the hadron spectrum.

% Given the literature I have read, the currents are actually re-expressed in terms of Vector and Axial currents as:

% \begin{eqnarray} 
% V^{{\mu},a} = R^{{\mu},a} + L^{{\mu},a} = \bar{q}{\gamma}^{\mu}\frac{{\lambda}^{a}}{2}{q} \label{eq:V currents}\\
% A^{{\mu},a} = R^{{\mu},a} - L^{{\mu},a} = \bar{q}{\gamma}^{\mu}{\gamma}_{5}\frac{{\lambda}^{a}}{2}{q} \label{eq:A currents}
% \end{eqnarray}

% The reason this could be is, because now, comparing the current algebras in \ref{eq:L currents} \& \ref{eq:R currents} with those of \ref{eq:V currents} \& \ref{eq:A currents}, the symmetry of lagrangian can be rexpressed as:

% \[SU(3)_{L} \otimes SU(3)_{R} \rightarrow SU(3)_{V} \otimes SU(3)_{A} \]

% and when discussing the repurcussions of this symmetry, instead of $q_L$ and $q_R$, we can use

% \[q = \begin{pmatrix} q_{u}\\ q_{d}\\ q_{s} \end{pmatrix} \]

% Note that from the reexpressed of current algebra of \ref{eq:V currents} \& \ref{eq:A currents}, one sees that the Axial current gets an additional ${\gamma}_{5}$ matrix such that rotations in the $SU(3)_A$ can be approximated as:

% \[ ~ e^{-{\lambda}^{a}{\theta}_{a}{\gamma}_{5}} \]

% [My conjectures above may not be right \& hence all that follows (that will already include a lot of 'vague' mathematics to get to the point of 'parity doubling') may not be the correct way to look at things]

% We can now proceed and construct irreducible representations from the fundamental one. Since we are using the triplet of flavor states as our fundamental representation, we can do ahead use the familiar meson octet formed from 

% \[ q\bar{q} states:  3 \otimes \bar{3} = 8 \oplus 1 \]

% Consider rotations under the $SU(3)_A$ group. Putting aside the ${\gamma}_{5}$ in the transformation, the usual transformations

% \[ ~ e^{-{\lambda}^{a}{\theta}_{a}} \]

% will contain matrices using which we can transform each member of the multiplet into any other one. All the members have the same mass i.e. degenerate eigenstates of the hamiltonian. (Actually this is just what we expect from the $SU(3)_V$ part of the symmetry)

% The ${\gamma}_{5}$, however brings about a difference. Consider a particular member of the meson octet:

% \[ u\bar{d}: \pi^{+} \]

% If there were no ${\gamma}_{5}$, we could construct appropriate matrices and reach any of the other members of the multiplet i.e.

% \[\pi^{-/0},  K^{+/-/0}, \bar{K}, \eta  \]

% (Again, this is precisely what happens under $SU(3)_V$).

% But now consider the action of $e^{{\lambda}^{a}{\theta}_{a}{\gamma}_{5}}$, where the ${\gamma}_{5}$ acts on each of the dirac fields in the multiplet. Under such a rotation:

% \begin{eqnarray} \label{eq:LR transform}
% q_R \rightarrow q_R' = e^{i{\lambda}^{a}{\theta}_{a}}q_R ; \bar{q}_R \rightarrow \bar{q}_R' = \bar{q}_Re^{-i{\lambda}^{a}{\theta}_{a}} \\
% q_L \rightarrow q_L' = e^{-i{\lambda}^{a}{\theta}_{a}}q_L ; \bar{q}_L \rightarrow \bar{q}_L' = \bar{q}_Le^{i{\lambda}^{a}{\theta}_{a}}
% \end{eqnarray}

% where I have made use of the fact that ${\gamma}_{5}$ is the Chirality operator with positive and negative eigenstates:

% \[ {\gamma}_{5}q_R = q_R \]
% \[ {\gamma}_{5}q_L = -q_L \]

% Rexpressing the pion as:

% \[ u\bar{d} = u_L\bar{d}_R + u_R\bar{d}_L\]

% Under $SU(3)_A$ using \ref {eq:LR transform}

% \begin{equation} \label{eq:A transform}
% {u\bar{d}}^{'} = e^{-i{\lambda}^{a}{\theta}_{a}}u_L\bar{d}_Re^{-i{\lambda}^{a}{\theta}_{a}} + e^{i{\lambda}^{a}{\theta}_{a}}u_R\bar{d}_Le^{i{\lambda}^{a}{\theta}_{a}}
% \end{equation}

% Under $SU(3)_V$ 
% \begin{equation} \label{eq:V transform}
% {u\bar{d}}^{'} = e^{i{\lambda}^{a}{\theta}_{a}}u_L\bar{d}_Re^{i{\lambda}^{a}{\theta}_{a}} + e^{i{\lambda}^{a}{\theta}_{a}}u_R\bar{d}_Le^{i{\lambda}^{a}{\theta}_{a}}
% \end{equation}

% Comparing \ref{eq:A transform} \& \ref{eq:V transform}, it can be argued (and it is indeed provied more rigorously) that due to negative sign in the exponential for the first term in \ref{eq:A transform}, there is going to be relative sign difference between the two expressions for the transformation of the pion under $SU(3)_A$ vis-a-vis $SU(3)_V$. Under parity, where left handed fields turn into the right handed ones and vicevera, the transformed state under $SU(3)_A$, will therefore become a negative parity eigenstate. Thus, as a result of this symmetry, for every positive parity object in a multiplet, there should be another object in a similar dimensional multiplet with negative parity. 

% [Give example of not finding such a "doubling" for the baryon octet; Can it also be argued that if this symmetry held then the pseudoscalar \& vector meson octet should have the had the same mass? The fact that the pseudoscalars have lower mass compared to their 'parity doubled' vector meson partners, can also be 'symptoms' of a spontaneously broken symmetry. I need to try articulate this idea better relating the pseudoscalar octet to the Goldstone bosons]

% \[SU(3)_V \otimes SU(3)_A \rightarrow SU(3)_V\]

% $SU(3)_V$ is the remnant symmetry (approximate), that is used for the basis of the Quark Model and approximate degenerate multiplets can be found. $SU(3)_A$ part of the symmetry appears to be broken.

% \subsection{Summary}
% To summarize, we have identified one local, $SU(3)_{color}$, and three global symmetries, $U(1)_V, SU(3)_V, SU(3)_L \times SU(3)_R$ of the lagrangian. 

% \begin{enumerate}
% \item $SU(3)_{color}$ is an exact local symmetry, the requirement of which brought about interactions of gauge bosons with  colored quark fields and among themselves [the latter is a distiguishing feature of this theory]
% \item $U(1)_V$ is an exact global symmetry. I did not go into details of this symmetry, but it requires that all the left and right handed fields be transformed by the \textit{same} phase \& that the mass matrix be diagonal (does not require equal quark masses). It is easy to verfiy this symmetry by applying the following transformation to the lagrangian of \ref{eq:Chiral lagrangian}\\

% All left handed fields transform by
% \[ e^{-i\theta_L}\]
% All right handed fields by
% \[ e^{-i\theta_R}\]

% with $\theta_L$ = $\theta_R$

% This condition yields conserved currents $\bar{u}{\gamma}_{\mu}u$, $\bar{d}{\gamma}_{\mu}d$, $\bar{s}{\gamma}_{\mu}s$. Again, we can associate a 'conserved charge' with these currents, which will commute with the Hamiltonian and hence be a conserved quantity. In this case the 'conserved charge' is associated with the conserved Baryon number. 

% \item $SU(3)_V$ is a global symmetry that requires all quark masses to be equal. It is only approximately realized, which is attributed to the fact that quark masses are infact not equal (and in turn the hadron multiplets do not have the same mass)

% \item $SU(3)_V \otimes SU(3)_A$ is a global symmetry that requires quark masses to be zero. The $SU(3)_A$ part of the symmetry appears to be broken and can be attributed to the fact that quark masses are infact not zero.
% \end{enumerate}

% So far, we have only made use of (or rather after experimental observation, used the imposed) global symmetries of the lagrangian to accomodate and organize the degrees of freedom that we observed (hadrons). The Eightfold way provides an organizational structure that uses the $SU(3)_V$ approximate symmetry. Any disparities in masses within a multiplet is ascribed to the fact that masses are infact not equal [statements like this need rigorous verification].

% $SU(3)_V \otimes SU(3)_A$ is broken [I am reserving the word spontaneous for now till I am comfortable with expressing it]. This helps in explaining differences of masses between multiplets [pseudoscaler meson \& vector meson octets for example?].

% All our conjectures so far have used the fundamental lagrangian. To accomodate the Quark Model, an approximate $SU(3)_{flavour}$ symmetry was imposed. Another global symmetry was imposed, that approximated the bare quark masses to be zero, which is a valid approximation considering that the observed hadron spectum is heavier. This is the Chiral Symmetry of QCD, the breaking of which explains the disparaties in masses of the hadron spectrum [again, such statements should be stated with proofs)

% [I can see how Chiral Symmetry might have come about, since requiring it explains the presence of 'parity partners', which we sometimes observe and don't know why their masses are different. However, I would expect it to break, since the bare quark masses would acquire additional mass due to self couplings - but perhaps, like in QED, it was expected that the mass shift due to self energy consideration would not be much different from the bare mass scale, hence the symmetry would continue to hold.]

% Given that we have some kind of organization of the hadron spectrum in terms of 'valence/constituent' quarks, can we understand these bound states like we do a hydrogen atom i.e. in terms of d.o.f of the QCD lagrangian? To study the hydrogen atom, we begin with an understanding of QED, having observed and renormalized the electron mass and all it's EM interactions in our 'real' world energy regime. However, a similar study of hadrons, is not that simple. 

% Not only are quarks always 'Confined', the study of the interactions of QCD cannot be carried out perturbatively.  As a result, vis-a-vis the electron, we have not observed a quark in its asymptotically free state. Additionally, due to the non perturbative nature of the theory, it is not trivial to map the evolution of the quark mass. In QED, we can can begin with a Lagrangian at a given energy range, and put in the meaured mass and the EM coupling, which are essentially parameters, into the lagrangian. We can begin QED at a low energy scale (real life) and carry forward the theory to higher and higher energy ranges, mapping the evolution of the mass and coupling. 

% In QCD, let's say we begin at the opposite energy spectrum, where the quarks are 'free', hence the strong coupling is close to being zero. So, we have an idea as to what to use as the parameter for the strong coupling from experience [be more precise about this experience of having observed quarks as asymptotically free]. But what do we use for the mass of the quarks?[give 'speculative' numbers for the bare quark mass]. 

% Even if we have somehow obtain parameters of QCD to plug into the lagrangian at high energies, how do the degrees of freedom and their couplings evolve to real world energy levels?
% The mass of the quarks clearly has repurcussions for the global symmetries we have been using so far, mainly for the Chiral Symmetry. Due to self interactions, particles get heavier [in the sense the 'exact' propagator has a pole at a new physical mass at different energy scales; be more precise here]. In the case of electrons, the physical mass does not change by much at different energy levels. But what about the quarks?

% We begin with bare quark masses of roughly 7.5 MeV, 4.2 MeV and 150 MeV for the u,d \& s quarks respectively [how are these numbers obtained]. According to the quark model, the proton's quantum number comes from it's constituent/valence quarks i.e. 2u and 1d quarks [look over contributions of sea quark to mass?] Then roughly the mass of each quark in this energy regime is 350 MeV. The approximate $SU(3)_{flavour}$ continues to hold in this sense [equal masses of quarks], but what about chiral symmetry? We already begain with a slightly broken chiral symmetry, but now the masses are much larger [or atleast as inferred from the Quark Model] [What about if the quarks continued to be massless, but their potential just increased linearly or in some other fashion, giving the hadron it's huge mass - but then, if this was so, Chiral Symmetry would not be broken?]

% A lot of the questions could be put to rest, if we could use the exact local symmetry of the lagrangian i.e. $SU(3)_{color}$ But this is not trivial. At every energy level, for the theory to be rigourous, we have to understand the manifestations of its d.o.f and their interactions. Because of Confinement, we have to study them in their bound states.

% [Aside Note: Look into relative contribution of each valence quark to the hadron's composite quantum numbers]. 


% \begin{center}
% \line(1,0){250}
% \end{center}

% I intend to use the above material to motivate my research, and use it make a couple of introductory slides. As a logical continuation, from this point i.e. after motivating the research:
% \begin{enumerate}
% \item I am going to introduce the idea of using virutal photons as probes of varying spatial resolution (varying $q^{2}$) and hence probing nucleons and their constituents at various energy scales, to provide data to test our models \& theories of QCD, to see how what we beleive to be the fundamental d.o.f manifest themselves at different 'energy scales'. 
% \item Introduce electrocouplings, and how combining experimental results of analysis from various channels, can be used to compare predictions from non-perturbative QCD and model based calculations about the manifestations of the QCD d.o.f
% \end{enumerate}


\end{document}
