\documentclass{article}
\usepackage{graphicx}
\usepackage{amssymb,amsmath}
\usepackage{hyperref}
\usepackage{color}
\usepackage[sharp]{easylist} 

%User defined "stuff"

%NOTE, I did not use the preferred
%\verb command, since it did not seem to work inside \newcommand
\newcommand{\code}[1]{\texttt{#1}}

\newcommand{\bi}{\begin{itemize}}
\newcommand{\ei}{\end{itemize}}

\newcommand{\be}{\begin{enumerate}}
\newcommand{\ee}{\end{enumerate}}

\definecolor{darkgreen}{rgb}{0,0.6,0}
\newcommand{\atgr}[1]{\textcolor{darkgreen}{#1}}
\definecolor{orange}{cmyk}{0,0.6,1,0}
\newcommand{\ator}[1]{\textcolor{orange}{#1}}

\newcommand{\fb}{\paragraph{Feedback}}
%

\numberwithin{equation}{subsection}
\begin{document}

\tableofcontents

\section{Meeting minutes: 11-20-13}
\subsection{Formalism}
Using the form of the double-differential Cross-section for single-pion Electroproduction(2 d.o.f.), we can formalize the following for double-charged-pion Electroproduction:
\begin{eqnarray}
\left(\frac{d\sigma}{dX^{ij}d\phi^{j}}\right)^{h} = 
A^{ij} +  B^{ij}\cos\phi^{j} + C^{ij}\cos2\phi^{j} + hPD^{ij}\sin\phi^{j}
\end{eqnarray}
where
\begin{itemize}
	\item ij = index over Varset,Variable (3x5 matrix)
	\item $R2^{ij}_{\alpha} \doteq 
	[A^{ij},B^{ij},C^{ij},D^{ij}] \equiv 
	[R_{T}+\epsilon_{L}R_{L}, R_{LT}, R_{TT}, R_{LT'}]$
	\item $R2^{ij}_{\alpha} = f(Q^{2},W,X^{ij})$
\end{itemize}

For convenience, I define the following:
\begin{eqnarray}
f^{h}(X^{ij},\phi^{j}) \doteq \left(\frac{d\sigma}{dX^{ij}d\phi^{j}}\right)^{h}
\end{eqnarray}


\subsection{Event selection}
\begin{enumerate}
	\item \code{eid}
	\item \code{efid}
	\item \code{momcorr}
	\item \code{MM Cuts}
\end{enumerate}

\subsection{R2 Extraction method}
Of the methods listed earlier:
\begin{enumerate}
	\item Fit $f^{h}(X^{ij},\phi^{j})$ to extract `R2`
	\item Calculate Asymmetry $\doteq$ $f^{h=+}-f^{h=-}$ and then extract $D^{ij}$
	\item $\int f^{h}(X^{ij},\phi^{j}) * (\cos\phi/\cos 2\phi/\sin\phi)d\phi$ to extract $B^{ij}/C^{ij}/D^{ij}$
\end{enumerate}
Method 3. is used, which even at the level of algorithmic detail is listed below. \textcolor{red}{NOTE that when multiplying by $\sin\phi$, the sign of the polarization is explicity used}

For every \code{q2wbin}:
\begin{enumerate}
	\item \code{h5[pol]} where \code{pol} $\in$ \{POS,NEG,UNP,AVG\}; \code{pol} $\neq$ AVG
	\item \code{h5m[pol,pob]} = \code{h5[pol]}$\cdot$\code{h5f[pob]} 
		\begin{itemize}
		\item \code{pob} $\in$ \{A,B,C,D\}; \code{pol} $\neq$ AVG
		\item \code{h5f[pob]}:
			\begin{itemize}
			\item For every bin \code{i}, \code{h5f[pob](i)} = \code{f[pob](i)}
			\item \code{f[pob]} $\in$ \{N.A.,$\cos\phi$,$\cos 2\phi$,$\color{red}{\text{sign(pol)}}$ $\sin\phi$\}
			\end{itemize}
		\end{itemize}
	\item \code{hR2\_Xij[pol,pob]} = \code{h5m[pol,pob]} \code{Project} on to $X^{ij}$; \code{pol} $\neq$ AVG
	\item \code{hR2\_Xij[pol=AVG,pob]} = (\code{hR2\_Xij[pol=POS,pob]} + \code{hR2\_Xij[pol=NEG,pob]})/2
\end{enumerate}

\subsection{Observations}
\bi
	\item Focussed only on \code{<B/C/D>\_1THETA}
	\item \code{Top 1:2:3:4} used
\ei

\paragraph{\textbf{Consistencies(C)}:}
\be
	\item \atgr{\code{<B/C>[pos]}=\code{<B/C>[neg]}=\code{<B/C>[unp]}}
	\item \ator{\code{EF-C[unp]} $\approx$ \code{SF-C[unp]}}
		\fb
		To ensure that this consistency is not due to \code{Hole-Filling}, see how well \code{EC-C[unp]} agrees with \code{SF-C[unp]}
\ee

\paragraph{\textbf{Inconsistencies(I)}:}
\be
   	\item \ator{\code{EF-D[unp]} $\neq$ 0}
   	\be
   		\item \ator{\code{D[pos]} = \code{-D[neg]}}
   		\item \ator{\code{D[unp]} = \code{D[pos]}}
   	\ee
   	\fb
   	These inconsistencies may be resolved if there is an additional $\sin\phi$ dependence present:\\
   	$f^{h}(X^{ij},\phi^{j}) \rightarrow f^{h}(X^{ij},\phi^{j}) + X\sin\phi$
  	

   	\item \code{SF-D[unp]} $\neq$ 0
   	\be
   		\item \code{SF-D[unp]} $\neq$ \code{EF-D[unp]}
   	\ee
   	\item \code{SF-B[unp]} $\neq$ \code{EF-B[unp]}
\ee


\end{document}
